\documentclass{article}
\usepackage{quoting}
\usepackage{datetime}
\usepackage[utf8]{inputenc}
\usepackage{graphicx}
\setcounter{secnumdepth}{4}
\date{\today \\ \currenttime} 
\title{Sistemas Distribuídos}
\author{Rui Raposo}


\begin{document}
\maketitle

\section{Modelos Fundamentais}

Os sistemas distribuídos podem ainda ser analisados segundo \textbf{3 aspetos travensais a todos os sistemas}:

\begin{itemize}
	\item Modelo de Interação (ou de sincronismo);
	\item Modelo de Falhas (ou avarias);
	\item Modelo de Segurança.
\end{itemize}

\subsection{Modelo de Interação}

Interação é a ação (comunicação e sincronização) entre as partes para realizar um qualquer trabalho.

É afetada por \textbf{dois} aspetos:
\begin{enumerate}
	\item Performance dos canais de comunicação;
	\item Inexistência de um tempo global.
\end{enumerate}

\subsubsection{Performance dos canais de comunicação}

\begin{itemize}
	\item \textbf{Latência} -- Intervalo de tempo que medeia entre o início da transmissão de uma mensagem por um processo e o início da sua receção pelo outro processo.
	\\
	Depende de :
		\begin{itemize}
			\item Tempo requerido pelo sistema operativo em ambos os lados da comunicação;
			\item Demora no acesso aos recursos da rede;
			\item Demora (\textit{delay}) de transmissão pela rede.
		\end{itemize}
	\item Largura de banda (\textit{bandwidth}) -- Total de informação que pode ser transmistida pela rede num dado intervalo de tempo;
	\item Jitter -- Variação no tempo necessário para enviar grupos de mensagens consecutivos constituintes de uma informação transmitida de um ponto para outro na rede (\textbf{importante na transmissão de som e imagem}).
\end{itemize}

\subsubsection{Inexistência de um tempo global}

\begin{itemize}
	\item Cada computador tem um relógio interno;
	\item Cada relógio tem um \textbf{\textit{drift}} (um desvio) do \textbf{tempo de referência};
	\item Os \textit{drifs} de dois relógios distintos são também distintos (o que significa que entre eles o tempo será sempre divergente).
\end{itemize}

\textbf{Uma solução passa por obter o tempo fornecido por GPS e enviar aos participantes do sistema distribuído}, mas existe um \textbf{problema} : o envio dessa mensagem!

Duas variantes no modelo de interação:

\paragraph{Sistemas distribuídos \textbf{síncronos}}. \\ 

Sistemas onde podem existir limites máximos de tempo conhecidos para:

\begin{itemize}
	\item Tempos de execução dos processos; 
	\item Atrasos na comunicação;
	\item Variação;
	\item O tempo necessário para executar cada passo de um processo tem um limite inferior e um limite superior conhecidos;
	\item Cada mensagem transmitida por um canal é recebida dentro de um limite de tempo conhecido;
	\item Cada processo tem um relógio cujo desvio máximo para o tempo de referência é conhecido.
\end{itemize}

\textbf{Podem definir-se \textit{timeouts} para detetar falhas}.

Dificuldade em contrar os limites para os tempos, mais difícil ainda, provar a sua correção.

\paragraph{Sistemas distribuídos \textbf{assíncronos}}. \\ Não possui limites para:

\begin{itemize}
	\item Tempo de execução dos processos -- cada passo de execução pode levar um tempo arbitrariamente longo;
	\item Tempo de transmissão de mensagems -- uma mensagem pode chegar rapidamente ou demorar dias;
	\item O desvio para o tempo de referência pode ser um qualquer;
\end{itemize} 

Exemplos de um sistema assíncrono: Internet.

Como lidar com longos tempos de espera:

\begin{itemize}
	\item O sistema pode avisar o utilizador que o tempo de espera pode ser longo e solicitar uma alternativa;
	\item O sistema pode dar oportunidade ao utilizador para fazer outras coisas;
\end{itemize}

Este tipo de sistema, \textbf{pode levar a problemas da ordenação de eventos}.

\subsection{Modelo de Avarias}

Uma avaria é qualquer alteração do comportamento do sistema em relação ao esperado.

Avarias podem acontecer e atingir \textbf{processos ou canais de comunicação}.

\subsubsection{Tipos de Avarias}

\begin{itemize}
	\item Avarias por omissão;
	\item Avarias arbitrárias;
	\item Avarias em tempo.
\end{itemize}

\paragraph{Avarias por omissão.}

\begin{itemize}
	\item Quando um processo deixa de funcionar em algum ponto do sistema distribuído;
	\item Quando o canal de comunicação falha.
\end{itemize}

\textbf{Tipos de avarias por omissão:}

\begin{itemize}
	\item \textit{Fail-stop} -- o processo bloqueou (\textit{crashed}) e esse facto pôde ser detetado por outros processos.
	\\
	\textit{Crash} -- o processo aparentemente bloqueou, mas não é possível garantir que apenas deixou de responder por estar muito lento, ou porque as mensagens que enviou não chegaram;
	\item \textit{Omission} -- uma mensagem colocada no buffer de emissão nunca chega ao buffer de recepção (pode ocorrer por falta de espaço no buffer);
	\item \textit{Send-omission} -- uma mensagem perde-se entre o emissor e o buffer de emissão;
	\item \textit{Receive-omission} -- uma mensagem perde-se entre o buffer de receção e o recetor.
\end{itemize}


\paragraph{Avarias arbitrárias.}

\begin{itemize}
	\item Qualquer tipo de erro pode aparecer:
	\begin{enumerate}
		\item Nos processos:
			\begin{itemize}
				\item Processo não responde;
				\item Estado do processo é corrompido;
				\item Responde de forma errada;
				\item Responde fora de tempo.
			\end{itemize}
		\item Nos canais de comunicação.
			\begin{itemize}
				\item Mensagens corrompidas; 
				\item Mensagens não entregues;
				\item Mensagens duplicadas;
				\item Mensagens inexistentes são entregues.
			\end{itemize}
	\end{enumerate}
\end{itemize}

\textbf{São raras de ocorrer nos canais de comunicação porque o software de comunicação protege as mensagens com somas de verificação (\textit{checksums}), números de sequenciamento, etc}.

\paragraph{Avarias em tempo.}

\begin{itemize}
	\item Ocorrem quando o tempo limite para um evento ocorrer é ultrapassado;
	\item Em sistemas eminentemente síncronos é um indicativo seguro de falha;
\end{itemize}

\subsection{Modelo de Segurança}

\begin{itemize}
	\item Proteção das entidades do sistema, processo/utilizador;
	\item \textbf{Direitos de acesso} especificam que \textbf{entidades} podem aceder, e de \textbf{que forma, a que recursos}.
	\item O servidor é responsável por verificar a \textbf{identidade} de quem fez o pedido, e verificar se essa entidade tem \textbf{direitos de acesso} para realizar a \textbf{operação pretendida};
	\item O cliente deverá verifica a identidade de \textbf{quem lhe enviou a resposta}, para ver se a resposta veio da entidade esperada.
\end{itemize}

\textbf{Que ameaças?}

Suponde que existe um processo inimigo capaz de:

\begin{itemize}
	\item Enviar qualquer mensagem para qualquer processo;
	\item Intercetar (ler/copiar) qualquer mensagem trocada entre 2 processos.
\end{itemize}

Classificação das ameaças:

\begin{itemize}
	\item Aos processos;
	\item À comunicação;
	\item Negação de serviço.
\end{itemize}

\subsubsection{Ataques a processos}

\begin{itemize}
	\item Ao projetar um servidor, ter consciência de que:
	\begin{itemize}
		\item Os protocolos de rede não oferecem proteção para que o servidor saiba a identidade do emissor (IP inclui o endereço do computador origem da mensagem, mas um processo inimigo pode forjar esse endereço);
		\item Um cliente também não dispõe de métodos para validar as respostas de um servidor;
	\end{itemize}
\end{itemize}

\subsubsection{Ataques a canais de comunicação}

\begin{itemize}
	\item Um processo inimigo pode copiar, alterar ou injetar mensagens numa rede;
	\item A comunicação pode ser violada por processos que observam a rede à procura de mensagens significativas (essas mensagens podem posteriormente ser reveladas a terceiros).
\end{itemize}

\subsubsection{Negação de Serviço}

Um processo intruso \textbf{captura} uma mensagem de solicitação de serviço e \textbf{retransmite-a} inúmeras vezes ao destinatário, fazendo-o executar sistematicamente o mesmo serviço e \textbf{ultrapassando} a sua capacidade de resposta. Como lidar com estas ameaças? \textbf{Utilização de canais seguros}.


\textbf{Canal Seguro} -- Canal utilizado para comunicação entre dois processos com as seguintes características:

\begin{itemize}
	\item Cada processo pode identificar com 100 por cento de confiança a entidade responsável pela execução de outro processo;
	\item As mensagens que são transferidas de um processo para outro são garantidas do ponto de vista da integridade e da privacidade;
	\item As mensagens têm garantia de não repetibilidade ou reenvio por ordem distinta (cada mensagem inclui um tempo físico ou lógico).
\end{itemize}


\end{document}