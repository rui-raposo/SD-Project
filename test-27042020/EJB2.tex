\documentclass{article}
\usepackage{quoting}
\usepackage{datetime}
\usepackage[utf8]{inputenc}
\usepackage{graphicx}
\setcounter{secnumdepth}{4}
\date{\today \\ \currenttime} 
\usepackage{listings}


\title{EJB2}
\author{António Sardinha}
\date{May 2020}

\begin{document}

\maketitle

\section{Ciclo de vida de um EJB}
\begin{itemize}
    \item \textit{\textbf{Stateful Session Bean}}
    \begin{itemize}
        \item O cliente obtém a referência para o Bean (\textit{Create})
        \item O Bean é removido da memória para o disco (\textit{Passive})
        \item Único método invocado pelo utilizador (\textit{Remove})
    \end{itemize}
    \item \textit{\textbf{Stateless Session Bean}}\\
    O EJB \textit{container} cria uma pool de \textit{stateless beans}
    \begin{itemize}
        \item Para cada instância:
        \begin{enumerate}
            \item \textit{Dependency injection, if any}
            \item \textit{PostConstruct callback, if any}
        \end{enumerate}
    \end{itemize}
    \item \textit{\textbf{Singleton Session Bean}}
    \begin{itemize}
        \item Se o \textit{Singleton bean} tem a anotação \textit{@Startup}, a única instância do Bean é iniciada pelo \textit{container} quando é feito o \textit{\underline{deploy}} da aplicação.
        \item Possui os mesmos estados que um \textit{Stateless bean}.
    \end{itemize}
    \item \textit{\textbf{Message Driven Bean}}
    \begin{itemize}
        \item O EJB \textit{container} criar uma pool de MDBs.
    \end{itemize}
\end{itemize}

\section{\textit{Timers}}

\begin{itemize}
    \item O serviço de Timer do contentor de JEBs permite:
    \begin{itemize}
        \item Programar no tempo notificações para todos os tipos de EJB com \textbf{exceção de \textit{Stateful}} \textit{Session Beans}
    \end{itemize}
    \item São possíveis notificações que ocorrem:
    \begin{itemize}
        \item De acordo com um determinado calendário
        \item Num tempo específico (e.g.,12 de Setembro, 9:00 a.m.)
        \item Após um certo período de tempo (e.g., dentro de 4 dias)
        \item Em intervalos de tempo (e.g., cada 3 minutos)
    \end{itemize}
    \item Os \textit{Timers} podem ser programados ou automáticos
    \item \textit{Timers} programados são criados por invocação de um dos métodos da interdace \textit{TimerService}
    \begin{itemize}
        \item Quando expira o tempo de um \textit{timer} programado é executado o método anotado com @Timeout 
    \end{itemize}
    \item Criar um Timer programado.(1)
    \item Criar um Timer programado.(2)
    \item Criar um Timer programado.(3)
    Por omissão os \textit{Timers} são persistentes.
    \begin{itemize}
        \item Se o servidor falha, o timer fica guardado em memória persistente e reativado quando o servidor recupera.
        \item Se o timer expirar enquanto o servidor estiver inativo, o método @Timeout é invocado, após o restart do servidor.
    \end{itemize}
\end{itemize}

\begin{lstlisting}

    //Criação de um timer programado (1)
    long duration = 60000;
    Timer timer = timerService.createSingleActionTimer
    (duration, new TimerConfig());
    SimpleDateFormatter formatter = new SimpleDateFormatter
    ("MM/dd/yyyy ’at’ HH:mm");
    Date date = formatter.parse(“5/05/2020 at 13:00");
    
    Timer timer = timerService.createSingleActionTimer(date, 
    new TimerConfig());
\end{lstlisting}

\begin{lstlisting}

    //Criação de um timer programado (2)
    ScheduleExpression schedule = new ScheduleExpression();
    schedule.dayOfWeek("Mon");
    schedule.hour("12-17, 23");
    Timer timer = timerService.createCalendarTimer(schedule);
\end{lstlisting}

\section{EJB Timers}
\begin{itemize}
    \item Criar um Timer automático (1)
    \begin{itemize}
        \item Timers automáticos são criados após o \textit{deploy} de um \textit{Bean} que contém um método anotado com java.ejb.Schedule ou java.ejb.Schedules
        \item Um \textit{bean} pode ter vários timers automáticos (os timers programados são únicos por bean)
        \item Um método anotado com @Schedule funciona como um método de \textit{timeout} para o calendário especificado nos atributos da anotação. 
    \end{itemize}
    \item Criar um Timer automático (2)
    \begin{itemize}
        \item Usar \textit{Schedules} para especificar vários calendários para o mesmo método. 
    \end{itemize}
    \item Especificação de Intervalos
    \begin{itemize}
        \item Numa expressão da forma $x/y$, x o ponto de partida e y o intervalo.
        \item O \textit{wildcard} (*) pode ser usado na posição x e equivale a x = 0.
    \end{itemize}
\end{itemize}

\begin{lstlisting}
    //Criacao de um Timer automático
    @Schedules ({
    @Schedule(dayOfMonth="Last"),
    @Schedule(dayOfWeek="Fri", hour="23")
    })
    public void doPeriodicCleanup() { ... }
\end{lstlisting}

\begin{lstlisting}
    //Criacao de um Timer automático
    minute="*/10“; // significa, todos os 10 minutos começando às 0 horas.
    hour="12/2“; // significa todas as duas horas a começar ao meio-dia.
\end{lstlisting}

\end{document}
