\documentclass{article}
\usepackage{quoting}
\usepackage{datetime}
\usepackage[utf8]{inputenc}
\usepackage{graphicx}
\usepackage{listings}
\setcounter{secnumdepth}{4}
\date{\today \\ \currenttime} 
\title{Sistemas Distribuídos}
\author{Rui Raposo / António Sardinha}
\UseRawInputEncoding

\begin{document}
\maketitle

\section{Java Bean} -- Standard para definição de classes reutilizáveis

Classe Java que:
\begin{itemize}
	\item Tem todas as propriedades privadas;
	\item Métodos públicos (\textit{getters} e \textit{setters}) acedem às propriedades;
	\item Tem um construtor sem parâmetros;
	\item Implementa a interface \textit{Serializable}.
\end{itemize}

\textbf{Componente executada do lado do servidor que implementa a lógica da aplicação/negócio}.

O EJB \textit{container} oferece os serviços do sistema para:
\begin{itemize}
	\item Gestão de transações;
	\item Segurança;
	\item Concorrência;
	\item Escalabilidade.
\end{itemize}

\subsection{Tipos de EJB}

\subsubsection{\textit{Session Beans}}

\begin{itemize}
	\item Implementam um tarefa para a aplicação cliente;
	\item Opcionalmente podem implementar um \textit{web service}.
\end{itemize}

\begin{enumerate}
	\item \textit{Stateful};
	\item \textit{Stateless};
	\item \textit{Singleton}.
\end{enumerate}

\subsubsection{\textit{Message-Driven Beans}}

Atuam como \textit{listeners} para um determinado tipo de mensagens.

\subsection{Acesso a \textit{session Beans}}

O cliente de um \textit{session Bean} obtém a referência para uma instância do \textit{Bean} por:

\begin{itemize}
	\item \textit{Dependency injection};
	\item JNDI (\textit{Java Naming and Directory Interface}).
\end{itemize}

O cliente tem acesso a um \textit{session Bean} através dos métodos de uma interface ou através dos métodos públicos do \textit{Bean}.

\textbf{3 Tipos de acesso}:

\begin{itemize}
	\item Local;
	\item Remoto;
	\item Como \textit{web service}.
\end{itemize}

\begin{itemize}

\item \textbf{Local} -- contexto local (JVM) do servidor
\begin{lstlisting}[language=Java]
   @Local
   public interface ExampleLocal {...}
\end{lstlisting}

\item \textbf{Remote} -- interface remota (dentro ou fora da JVM)
\begin{lstlisting}[language=Java]
   @Remote
   public interface ExampleRemote {...}
\end{lstlisting}

\item \textbf{\textit{Web Service}} -- serviço sobre \textit{HTTP}
\begin{lstlisting}[language=Java]
   @WebService
   public interface InterfaceName {...}
\end{lstlisting}

\item \textbf{\textit{Dependency Injection}}
\begin{lstlisting}[language=Java]
@EJB
Example example
\end{lstlisting}

O habitual \textit{lookup} pode ser substituído pela anotação $@EJB$ em que o servidor $JEE$ implicitamente injeta o código para obter o EJB referenciado.

\end{itemize}

\subsection{\textit{Stateful Session Beans} -- Caraterísticas}

\begin{itemize}
	\item Cada instância de um Stateful Bean está associada a um único cliente;
	\item O Bean mantém o estado de sessão (não partilhado) com um dado cliente (estado = valores das variáveis de instância);
	\item Permitem guardar informação do cliente entre múltiplas invocações;
	\item Possuem um tempo de vida (configurável) até serem removidos.
\end{itemize}

\subsection{\textit{Stateless Session Beans} -- Caraterísticas}

\begin{itemize}
	\item Não mantêm informação específica de um cliente;
	\item O estado existe apenas durante a invocação de um método;
	\item O servidor gere uma \textit{pool} de instâncias que servem pedidos de vários clientes;
	\item Interface pode ser exposta como \textit{web service}.
\end{itemize}

\subsection{\textit{Singleton Session Beans} -- Caraterísticas}

\begin{itemize}
	\item Instanciado apenas \textbf{uma vez por aplicação};
	\item Estado partilhado entre \textbf{todos os clientes};
	\item Estado preservado entre invocações;
	\item Anotação $@Startup$ indica que deve ser instanciado no arranque da aplicação.
\end{itemize}

\section{\textit{Message Driven Beans}}

\subsection{\textit{Message Driven Beans} -- Caraterísticas}

\begin{itemize}
	\item Consiste num recetor \textbf{assíncrono} de mensagens Java;
	\item Não mantém estado;
	\item Um \textit{Meesage Driven Bean} pode processar mensagens de múltiplos clientes;
	\item O cliente não acede diretamente à interface do \textit{Message Driven Beans}, usa um serviço de mensagens:
		\begin{itemize}
			\item JMS -- \textit{Java Messaging Service};
		\end{itemize}
\end{itemize} 

\subsection{\textit{Message Driven Beans} -- Funcionamento}

\begin{itemize}
	\item As mensagens do cliente são enviadas para uma \textit{queue} Java e posteriormente entregues ao \textit{Message Driven Beans};
	\item O nomee da \textit{queue} é indicado na anotação $@MessageDriven$ do \textit{Message Driven Beans} que se torna automaticamente (sem código adicional) num consumidor;
\end{itemize}

\end{document}