\documentclass{article}
\usepackage{quoting}
\usepackage{datetime}
\usepackage[utf8]{inputenc}
\usepackage{graphicx}
\date{\today \\ \currenttime} 
\title{Sistemas Distribuídos}
\author{Rui Raposo}


\begin{document}
\maketitle

%Modelos de Programação Distribuída%
\section{Modelos de Programação Distribuída}

\begin{enumerate}
	\item Modelos de comunicação por mensagens (\textbf{comunicação por \textit{Sockets}});
	\item \textbf{Modelos Arquiteturais}.
\end{enumerate}

A arquitetura de um sistema distribuído é a estrutura do sistema em termos de \textbf{localização} das suas diferentes partes, do \textbf{papel} que cada parte desempenha e como se interrelacionam.

A arquitetura tem implicações no \textbf{desempenho, fiabilidade e segurança do sistema}.

%Modelos Arquiteturais%
\subsection{Modelos Arquiteturais}

Camadas de um sistema distribuído:
\begin{itemize}
	\item Aplicações e serviços;
	\item \textit{Middleware}, camada de \textit{software} que tem como objetivo mascarar a heterogeneidade de um sistema distribuído fornecendo um modelo de programação uniforme; 
	\item Sistema Operativo (\textbf{Plataforma});
	\item Computador e hardware da rede (\textbf{Plataforma}).
\end{itemize}

\subsubsection{Modelo -- Cliente Servidor}

Modelo independente do \textit{middleware} utilizado, \textbf{modelo assimétrico}.

\textbf{Servidor} : processo passivo que quando contactado por um cliente envia uma resposta.

\textbf{Cliente} : contacta o servidor com o objetivo de utilizar um serviço; envia um pedido e fica à espera da resposta.

\textbf{Cliente e servidor são papéis que podem ser desempenhados}. Uma entidade pode simultaneamente ser cliente e servidor. Um processo para responder a um pedido, pode ter que recorrer a outro serviço, sendo cliente deste.

Num sistema de informação típico, existem \textbf{três} classes de funcionalidades:

\begin{itemize}
	\item Camada de apresentação -- Responsável pela \textbf{\textit{interface}} com o utilizador;
	\item Camada de lógica de negócios -- Controlam o \textbf{comportamento} da aplicação;
	\item Camada de persistência de dados -- Assegura o \textbf{armazenamento e integridade dos dados}.
\end{itemize}

Servidor em \textbf{3 camadas}:
\begin{itemize}
	\item \textit{Web browser} -- HTML;
	\item Servidor \textit{Web} -- PHP, Java;
	\item Base de Dados -- MySQL, SQLServer, oracle;
\end{itemize}

\subsubsection{Cliente/Servidor \textbf{replicado}}

Existem vários servidores, capazes de responder aos mesmos pedidos.

\textbf{Vantagens}:
\begin{itemize}
	\item Permite distribuir a carga, melhorando o desempenho;
	\item Não existe um ponto de falha único.
\end{itemize}

\textbf{Problema}:
\begin{itemize}
	\item Manter o estado do servidor coerente em todas as réplicas.
\end{itemize}


\subsubsection{Cliente/Servidor \textbf{particionado}}

Existem vários servidores com a mesma \textit{interface}, cada um capaz de responder \textbf{a uma parte dos pedidos, DNS}.

Servidor redirige o cliente para outro servidor (\textbf{iterativo}).

Servidor invoca o pedido noutro servidor (\textbf{recursivo}).

\subsection{\textit{Proxy srvers and Caches}}

Quando um cliente necessita de um objeto, o serviço de \textit{caching} verifica se possui uma cópia atualizada do objeto, em caso afirmativo fornece essa cópia.

Uma \textit{cache} pode estar localizada no cliente ou em servidores \textit{proxy} que são partilhados por vários clientes.

Objetivo: \textbf{aumentar a disponbilidade e a performance do serviço}.

\subsection{Processos pares -- \textit{peer processes}}

\begin{itemize}
	\item Todos os processos desempenham papeis similares. Cada processo é responsável pela consistência dos seus dados (recursos) e pela sincronização das várias operações.
	\item Cada processo pode assumir (simultaneamente ou alternadamente) o papel de cliente e servidor do mesmo serviço.
	\item Paradigma de distribuição em que os serviços são suportados diretamente pelos seus clientes/utilizadores, sem recurso a uma infra-estrutura criada e mantida explicitamente para esse fim.
	\item A ideia base é conseguir explorar os recursos disponíveis nas máquinas ligadas em rede: cpu, disco, largura de banda...
	\item Modelos de interação e coordenação mais complexos (que em sistemas cliente/servidor).
	\item Algoritmos mais complexos.
	\item Não existe ponto único de falha.
	\item Grande potencial de escalabilidade.
	\item Apropriado para ambientes em que todos os participante querem cooperar para fornecer um dado serviço.
\end{itemize}

\subsubsection{Servidor de diretório centralizado}

\begin{itemize}
	\item Quando um novo processo se liga, informa o servidor central (do seu endereço, do seu conteúdo).
	\item A partir daí pode comunicar com os outros pares.
\end{itemize}	

\subsubsection{Serviço de diretório distribuído}

\begin{itemize}
	\item Quando um novo processo se liga, liga-se a um grupo.
	\item O líder do grupo regista o conteúdo de todos os elementos do grupo.
	\item Cada processo acede ao líder do seu grupo para localizar o que pretende.
	\item Cada líder pode aceder aos outros líderes.
\end{itemize}

\end{document}