\documentclass{article}
\usepackage{quoting}
\usepackage{datetime}
\usepackage[utf8]{inputenc}
\usepackage{graphicx}
\setcounter{secnumdepth}{4}
\date{\today \\ \currenttime} 
\title{Sistemas Distribuídos}
\author{Rui Raposo / António Sardinha}


\begin{document}
\maketitle

\section{Recordando...}

\begin{itemize}
	\item \textbf{Modelo de Avarias}: caracteriza o sistema em termos das falhas/avarias, i.e., dos desvios em relação ao comportamento especificado, que os seus componentes podem apresentar.
	\item \textbf{Modelo de Sincronismo}: caracteriza o sistema em termos do comportamento temporal dos seus componentes:
		\begin{itemize}
			\item Processos;
			\item Relógios locais;
			\item Canais de comunicação;
			\item Síncrono;
			\item Assíncrono.
		\end{itemize}
		Um sistema diz-se \textbf{síncrono} se e só se:
		\begin{enumerate}
			\item É conhecido um limite inferior e um limite superior para o tempo que cada processo leva a executar;
			\item É conhecido um limite inferior e um limite superior para o atraso na comunicação entre os seus componentes;
			\item Cada processo tem um relógio local e é conhecido um limite superior para o desvio na sua taxa de incremento.
		\end{enumerate}
		Um sistema diz-se \textbf{assíncrono} se \textbf{nada se assume sobre o comportamento temporal do sistema}.
		\item \textbf{Dilema}: É relativamente fácil resolver problemas com sistemas síncronos, mas é extremamente difícil construir um sistema síncrono.
\end{itemize}

\section{Tempo e Relógios -- O papel do tempo}

\begin{itemize}
	\item Precisa de ser medido com elevada \textbf{precisão};
	\item Precisa de ser medido de forma \textbf{consistente} pelos diversos componentes de um sistema;
	\item Crucial na \textbf{ordenação} de eventos (observadores distintos podem testemunhar eventos por ordens diferentes);
	\item Tempo Real?
		\begin{itemize}
			\item Função monótona contínua e crescente;
			\item Unidade: segundo.
		\end{itemize}
\end{itemize}

O uso do tempo em sistemas distribuídos é feito em \textbf{dois} aspetos:

\begin{enumerate}
	\item Registar e observar a localização de eventos na \textit{timeline}. Queremos saber qual a sequência em que ocorreu um conjunto de eventos (possivelmente distribuídos por várias máquinas);
	\item Forçar o futuro posicionamento de eventos na \textit{timeline}. Sincronização do progresso concorrente do sistema para conhecermos qual a sequência de um conjunto de acontecimentos podemos marcar o instante de ocorrência atribuindo um, \textbf{\textit{Timestamp}}.
\end{enumerate}

\textit{\textbf{Timestamp}} -- sequència de caracteres que marcam a data e/ou tempo no qual um certo evento ocorreu (exemplo, data de criação/modificação de um ficheiro).

-- Um \textit{timestamp} está associado a um ponto na \textit{timeline}.

Se queremos comparar a duração de vários acontecimentos podemos usar:
\begin{itemize}
	\item \textbf{intervalos de tempo}, cadeia de tempo composta por vários intervalos adicionados;
	\item \textbf{\textit{Timers}/Relógios locais}, implementam a abstração da \textit{timeline}.
\end{itemize}

Num sistema distribuído cada evento pode ocorrer em diferentes locais, cada um com a sua \textit{timeline}. \textbf{Como conciliar diferentes \textit{timelines}? Como medir durações distribuídas?}

\begin{itemize}
	\item \textbf{Tempo Global}, implementa a abstração de um tempo universal, através de um relógio que fornece o mesmo tempo a \textbf{todos} os participantes no sistema;
	\item \textbf{Tempo Absoluto}, padrões universalmente ajustados, disponíveis como fontes de tempo externo para o qual qualquer relógio interno se pode sincronizar;
	\item \textbf{Relógio físico local}, o modo mais comum para fornecer uma fonte de tempo num processo.
	\begin{itemize}
	    \item Equipamento físico, que conta as oscilações que ocorrem num cristal de quartzo a uma dada frequência.
	    \item Cada oscilação do cristal decrementa o contador de uma unidade.
	    \item Quando o cotador chega a zero, é gerado um \textit{interrupt} e o contador é recarregado com o valor inicial.
	    \item Cada \textit{interrupt} é designado como um "\textit{clock tick}".
	\end{itemize}
	\item Um relógio num processo correto, \textit{k}, implementa uma função \textbf{discreta}, monótona crescente, $pc_k$, que mapeia o tempo real \textit{t} em tempo de relógio $pc_k(t)$.
	\item Problemas dos relógios físicos:
	    \begin{itemize}
	        \item \textbf{Granularidade:} relógios físicos são granulares, isto é, avançam uma unidade em cada\textit{tick}, $t_tk$:
	        
	        \[ pc_k ^ {tk+1} - pc_k ^ {tk} = g\]
	        \begin{itemize}
	            \item A frequência das oscilações varia com a temperatura.
	        \item Diferentes taxas de desvio em diferentes computadores.
	        \end{itemize}
	        \item \textbf{Taxa de desvio do relógio físico:} existe uma constante positiva $r_p$, a \textit{taxa de desvio (rate of drift)}, que depende não só da qualidade do relógio mas também das condições ambientais.
	        
	        \[ 0<= 1-r_p<= (pc_k(t_{tk+1}) - pc_k(t_{tk})) / g <= 1+r_p\]
	        \[ 0 <= t_{tk} <= t_{tk+1}\]
	    \end{itemize}
\end{itemize}
\section{\textit{Clock skew / Clock drift}}

\begin{itemize}
    \item \textbf{Skew} - é a diferença do valor do tempo lido de dois relógios diferentes.
    \item \textbf{Drift} - é a diferença no valor lido de um relógio e o valor do tempo fornecido por um relógio de referência perfeito por unidade de tempo do relógio de referência.
\end{itemize}
    \textbf{Ex.} drift de $10^{-5}$ segundos / segundo, significa que em cada segundo o relógio tem um desvio de 0,00001 segundos.
    
\section{Para que serve um \textbf{relógio local}?}

\begin{itemize}
    \item Fornecer \textit{timestamps} para eventos locais.
    \item Medir durações locais. (O erro causado pelo desvio é insignificante para durações pequenas.)
    \item Pode ser usado como um "\textit{timer}" para estabelecer \textit{timeouts}.
    \item Medir durações distribuídas \textit{round-trip}.
\end{itemize}

\section{\textbf{Relógios Globais} - Características}

\begin{itemize}
    \item Fornecer  o mesmo tempo para todos os intervenientes do sistema.
    \item \textit{Timestamping} de eventos distribuídos.
    \item Medição de durações distribuídas.
\end{itemize}

\section{\textbf{Relógios Globais} - Funcionamento}

\begin{itemize}
    \item É criado um relógio virtual $vc_p$ para cada processo $p$ a partir do relógio físico.
    \item É feita a sincronização de todos os relógios locais com o mesmo valor inicial $vc_p(t_{init})$
    \item Periodicamente os relógios virtuais são re-sincronizados -> Algoritmos de Sincronização 
\end{itemize}

\section{\textbf{Relógios Globais} - Propriedades}
\begin{itemize}
    \item \underline{Granularidade}: $g_v = vc_p(t_{k+1}) - vc_p(t_k)$
    \item \underline{Precisão} ($\pi_v$) : quão próximos os relógios se mantêm sincronizados entre si em qualquer instante do tempo. 
    \item \underline{Exatidão} (\textit{accuracy $-\alpha_v$}): quão próximos os relógios estão sincronizados em relação a uma referência de tempo real absoluto (sincronização externa)
\end{itemize}

\section{Sincronização interna vs Sincronização externa}

\begin{itemize}
    \item Sincronização interna:
    \begin{itemize}
        \item - relógios têm que obter precisão relativamente a um tempo interno ao sistema
    \end{itemize}
    \item Sincronização externa:
    \begin{itemize}
        \item - relógios tem que estar sincronizados com uma fonte externa de tempo universal
    \end{itemize}
\end{itemize}

\section{Referências de Tempo universal - Normas}

\begin{itemize}
    \item Tempo Atómico Internacional (TAI) -> função contínua monótona crescente a uma taxa constante.
    \item \textit{Universal Time, Coordinated (UTC)} -> referência de tempo política (Correção do TAI).
    \begin{itemize}
        \item \textbf{- Forma mais simples de obter o UTC}: por GPS - – é assegurada uma exactidão, em terra, <= 100ns para os relógios dos receptores de GPS.
    \end{itemize}

\end{itemize}

\section{Medição de durações \textit{round-trip}}

\begin{itemize}
    \item Certas durações distribuídas podem ser medidas sem a existência explícita de relógios globais.
    \item O atraso de entrega de uma mensagem pode ser calculado com um erro conhecido e limitado, se existir uma mensagem prévia recente no sentido inverso.
    \item Pré-requisitos para o uso deste método:
    \begin{itemize}
        \item Assegurar troca de mensagens frequente entre os sites relevantes
        \item Assegurar que o timestamping das transmissões de mensagens e entregas, também sejam trocados entre os sites relevantes
    \end{itemize}
\end{itemize}


\end{document}
