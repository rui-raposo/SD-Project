\documentclass{article}
\usepackage{quoting}
\usepackage{datetime}
\usepackage[utf8]{inputenc}
\usepackage{graphicx}
\setcounter{secnumdepth}{4}
\date{\today \\ \currenttime} 
\title{Sistemas Distribuídos}
\author{Rui Raposo}


\begin{document}
\maketitle

\section{Recordando...}

\begin{itemize}
	\item \textbf{Modelo de Avarias}: caracteriza o sistema em termos das falhas/avarias, i.e., dos desvios em relação ao comportamento especificado, que os seus componentes podem apresentar.
	\item \textbf{Modelo de Sincronismo}: caracteriza o sistema em termos do comportamento temporal dos seus componentes:
		\begin{itemize}
			\item Processos;
			\item Relógios locais;
			\item Canais de comunicação;
			\item Síncrono;
			\item Assíncrono.
		\end{itemize}
		Um sistema diz-se \textbf{síncrono} se e só se:
		\begin{enumerate}
			\item É conhecido um limite inferior e um limite superior para o tempo que cada processo leva a executar;
			\item É conhecido um limite inferior e um limite superior para o atraso na comunicação entre os seus componentes;
			\item Cada processo tem um relógio local e é conhecido um limite superior para o desvio na sua taxa de incremento.
		\end{enumerate}
		Um sistema diz-se \textbf{assíncrono} se \textbf{nada se assume sobre o comportamento temporal do sistema}.
		\item \textbf{Dilema}: É relativamente fácil resolver problemas com sistemas síncronos, mas é extremamente difícil construir um sistema síncrono.
\end{itemize}

\section{Tempo e Relógios -- O papel do tempo}

\begin{itemize}
	\item Precisa de ser medido com elevada \textbf{precisão};
	\item Precisa de ser medido de forma \textbf{consistente} pelos diversos componentes de um sistema;
	\item Crucial na \textbf{ordenação} de eventos (observadores distintos podem testemunhar eventos por ordens diferentes);
	\item Tempo Real?
		\begin{itemize}
			\item Função monótona contínua e crescente;
			\item Unidade: segundo.
		\end{itemize}
\end{itemize}

O uso do tempo em sistemas distribuídos é feito em \textbf{dois} aspetos:

\begin{enumerate}
	\item Registar e observar a localização de eventos na \textit{timeline}. Queremos saber qual a sequência em que ocorreu um conjunto de eventos (possivelmente distribuídos por várias máquinas);
	\item Forçar o futuro posicionamento de eventos na \textit{timeline}. Sincronização do progresso concorrente do sistema para conhecermos qual a sequência de um conjunto de acontecimentos podemos marcar o instante de ocorrência atribuindo um, \textbf{\textit{Timestamp}}.
\end{enumerate}

\textit{\textbf{Timestamp}} -- sequència de caracteres que marcam a data e/ou tempo no qual um certo evento ocorreu (exemplo, data de criação/modificação de um ficheiro).

-- Um \textit{timestamp} está associado a um ponto na \textit{timeline}.

Se queremos comparar a duração de vários acontecimentos podemos usar:
\begin{itemize}
	\item \textbf{intervalos de tempo}, cadeia de tempo composta por vários intervalos adicionados;
	\item \textbf{\textit{Timers}/Relógios locais}, implementam a abstração da \textit{timeline}.
\end{itemize}

Num sistema distribuído cada evento pode ocorrer em diferentes locais, cada um com a sua \textit{timeline}. \textbf{Como conciliar diferentes \textit{timelines}? Como medir durações distribuídas?}

\begin{itemize}
	\item \textbf{Tempo Global}, implementa a abstração de um tempo universal, através de um relógio que fornece o mesmo tempo a \textbf{todos} os participantes no sistema;
	\item \textbf{Tempo Absoluto}, padrões universalmente ajustados, disponíveis como fontes de tempo externo para o qual qualquer relógio interno se pode sincronizar;
	\item \textbf{Relógio físico local}, o modo mais comum para fornecer uma fonte de tempo num processo
\end{itemize}






\end{document}